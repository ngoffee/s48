\documentclass{article}
\usepackage{hyperlatex}


\newcommand{\cvar}[1]{\textrm{\textit{#1}}}

%%%%%%%%%%%%%%%% Latex prototypes
\texonly

\newcommand{\evalsto}{$\rightarrow$}

\newenvironment{protos}{\list{$\bullet$}
{\leftmargin1.2em\rightmargin0pt\itemsep0pt\parsep0pt\partopsep-2pt}}
{\endlist}

% The following is for prototypes that have return types.
%    (foo int int) -> int

\newcommand{\proto}[3]{%
\protonoindex{#1}{#2}{#3}%
\mainschindex{#1}}%

\newcommand{\protonoindex}[3]{\item\noindent\unskip%
\hbox{\spaceskip=0.5em\code{({#1}{\it#2\/})} {$\rightarrow$} {\it#3}}}

\newcommand{\cproto}[1]{\item\noindent\unskip%
\hbox{\spaceskip=0.5em\code{{#1}}}}

\newcommand{\cgcproto}[1]{\item\noindent\unskip%
\hbox{\spaceskip=0.5em\code{{#1}}}\hfill\penalty 0%
\hbox{ }\nobreak\hfill\hbox{\rm (may GC)}}

\newcommand{\protonoresultnoindex}[2]{\item\noindent\unskip%
\hbox{\spaceskip=0.5em\code{(\hbox{#1}{\it#2\/})}}}

\newcommand{\protonoresult}[2]{%
\protonoresultnoindex{#1}{#2}%
\mainschindex{#1}}%

\newcommand{\protoresult}[1]{\newline\unskip%
{\hspace*{2em}\code{{$\rightarrow$} {\it#1}}\hfill}}

% Syntax prototypes

\newcommand{\syntaxprotonoresultnoindex}[2]{\item\noindent\unskip%
\hbox{\spaceskip=0.5em\code{(\hbox{#1}{#2})}}\hfill\penalty 0%
\hbox{ }\nobreak\hfill\hbox{\rm syntax}}

\newcommand{\syntaxprotonoresult}[2]{%
\syntaxprotonoresultnoindex{#1}{#2}\mainschindex{#1}}

\newcommand{\syntaxproto}[3]{\item\noindent\unskip%
\hbox{\spaceskip=0.5em\code{(\hbox{#1}{#2})}} {$\rightarrow$} {\it#3}%
\hfill\penalty 0%
\hbox{ }\nobreak\hfill\hbox{\rm syntax}}

% This can be reduced

\newcommand{\pconstproto}[2]{\item\noindent\unskip%
\hbox{\spaceskip=0.5em\code{#1}}\hfill\penalty 0%
\hbox{ }\nobreak\hfill\hbox{\rm #2}}

% Variable prototype
\newcommand{\constproto}[2]{\pconstproto{#1}{#2}\mainschindex{#1}}

\newcommand{\constprotonoindex}[2]{\pconstproto{#1}{#2}}

\endtexonly

%%%%%%%%%%%%%%%% end of Latex proto definitions

%%%%%%%%%%%%%%%% HTML prototypes
\htmlonly

\newcommand{\evalsto}{\verb| => |}

\newenvironment{protos}{\begin{itemize}}{\end{itemize}}

% The following is for prototypes that have return types.
%    (foo int int) -> int

\newcommand{\protonoindex}[3]{%
%\cindex{\code{#1}}%
\item\noindent\code{({#1}{\var{#2}})~-->~{\var{#3}}}}

\newcommand{\proto}[3]{%
\protonoindex{#1}{#2}{#3}%
\mainschindex{#1}}%

\newcommand{\protonoresultnoindex}[2]{%
%\cindex{\code{#1}}%
\item\noindent\code{({#1}{\var{#2}\/})}}

\newcommand{\protonoresult}[2]{%
\protonoresultnoindex{#1}{#2}%
\mainschindex{#1}}%

\newcommand{\pconstproto}[2]{%
\item\noindent\code{{#1}}\prototag{#2}}

% Variable prototype
\newcommand{\constproto}[2]{\pconstproto{#1}{#2}\mainschindex{#1}}

\newcommand{\constprotonoindex}[2]{\pconstproto{#1}{#2}}

\newcommand{\cproto}[1]{%
\item\noindent\code{{#1}}}

\newcommand{\cgcproto}[1]{%
\item\noindent\code{{#1}}\prototag{may GC}}

\newcommand{\syntaxprotonoresult}[2]{%
\syntaxprotonoresultnoindex{#1}{#2}\mainschindex{#1}}

\newcommand{\syntaxprotonoresultnoindex}[2]{%
\item\noindent\code{({#1}{#2})}\prototag{syntax}}

\newcommand{\protoresult}[1]{%
\\{}\noindent\code{\ \ \ \ -->~{\it{#1}}}}

\newcommand{\syntaxproto}[3]{%
\item\noindent\code{({#1}{#2})~-->~{\var{#3}}}\prototag{syntax}}

\newcommand{\prototag}[1]{\hfill\hbox{(#1)}}

\endhtmlonly
%%%%%%%%%%%%%%%% end of HTML proto definitions


% Make a few big HTML files, and not a lot of small ones.
\setcounter{htmldepth}{1}

\makeindex

\title{Scheme 48 User's Guide}
\author{Richard A. Kelsey}

%\date{}

\begin{document}

\maketitle

\section{ASCII character encoding}

These are in the structure \code{ascii}.

\begin{protos}
\proto{char->ascii}{ char}{integer}
\proto{ascii->char}{ integer}{char}
\end{protos}
\noindent
These are identical to \code{char->integer} and \code{integer->char} except that
 they use the ASCII encoding.

\begin{protos}
\constproto{ascii-limit}{integer}
\constproto{ascii-whitespaces}{list of integers}
\end{protos}
\code{Ascii-limit} is one more than the largest value that \code{char->ascii}
 may return.
\code{Ascii-whitespaces} is a list of the ASCII values of whitespace characters
 (space, tab, line feed, form feed, and carriage return).

\section{Bitwise integer operations}

These functions use the two's-complement representation for integers.
There is no limit to the number of bits in an integer.
They are in the structures \code{bitwise} and \code{big-scheme}.

\begin{protos}
\proto{bitwise-and}{ integer integer}{integer}
\proto{bitwise-ior}{ integer integer}{integer}
\proto{bitwise-xor}{ integer integer}{integer}
\proto{bitwise-not}{ integer} {integer}
\end{protos}
\noindent
These perform various logical operations on integers on a bit-by-bit
basis. `\code{ior}' is inclusive OR and `\code{xor}' is exclusive OR.

\begin{protos}
\proto{arithmetic-shift}{ integer bit-count}{integer}
\end{protos}
\noindent Shifts the integer by the given bit count, which must be an integer,
 shifting left for positive counts and right for negative ones.
Shifting preserves the integer's sign.

\section{Arrays}

These are N-dimensional, zero-based arrays and
 are in the structure \code{arrays}.

The array interface is derived from one written by Alan Bawden.

\begin{protos}
\proto{make-array}{ value dimension$_0$ \ldots}{array}
\proto{array}{ dimensions element$_0$ \ldots}{array}
\proto{copy-array}{ array}{array}
\end{protos}
\noindent
\code{Make-array} makes a new array with the given dimensions, each of which
 must be a non-negative integer.
Every element is initially set to \cvar{value}.
\code{Array} Returns a new array with the given dimensions and elements.
\cvar{Dimensions} must be a list of non-negative integers, 
The number of elements should be the equal to the product of the
 dimensions.
The elements are stored in row-major order.
\begin{example}
(make-array 'a 2 3) \evalsto \{Array 2 3\}

(array '(2 3) 'a 'b 'c 'd 'e 'f)
    \evalsto \{Array 2 3\}
\end{example}

\code{Copy-array} returns a copy of \cvar{array}.
The copy is identical to the \cvar{array} but does not share storage with it.


\begin{protos}
\proto{array?}{ value}{boolean}
\end{protos}
\noindent
Returns \code{\#t} if \cvar{value} is an array.

\begin{protos}
\proto{array-ref}{ array index$_0$ \ldots}{value}
\protonoresult{array-set!}{ array value index$_0$ \ldots}
\proto{array->vector}{ array}{vector}
\proto{array-dimensions}{ array}{list}
\end{protos}
\noindent
\code{Array-ref} returns the specified array element and \code{array-set!}
 replaces the element with \cvar{value}.
\begin{example}
(let ((a (array '(2 3) 'a 'b 'c 'd 'e 'f)))
  (let ((x (array-ref a 0 1)))
    (array-set! a 'g 0 1)
    (list x (array-ref a 0 1))))
    \evalsto '(b g)
\end{example}

\code{Array->vector} returns a vector containing the elements of \cvar{array}
 in row-major order.
\code{Array-dimensions} returns the dimensions of
 the array as a list.

\begin{protos}
\proto{make-shared-array}{ array linear-map dimension$_0$ \ldots}{array}
\end{protos}
\noindent
\code{Make-shared-array} makes a new array that shares storage with \cvar{array}
 and uses \cvar{linear-map} to map indicies to elements.
\cvar{Linear-map} must accept as many arguments as the number of
 \cvar{dimension}s given and must return a list of non-negative integers
 that are valid indicies into \cvar{array}.
\begin{example}
(array-ref (make-shared-array a f i0 i1 ...)
           j0 j1 ...)
\end{example}
is equivalent to
\begin{example}
(apply array-ref a (f j0 j1 ...))
\end{example}

As an example, the following function makes the transpose of a two-dimensional
 array:
\begin{example}
(define (transpose array)
  (let ((dimensions (array-dimensions array)))
    (make-shared-array array
                       (lambda (x y)
		         (list y x))
		       (cadr dimensions)
		       (car dimensions))))

(array->vector
  (transpose
    (array '(2 3) 'a 'b 'c 'd 'e 'f)))
      \evalsto '(a d b e c f)
\end{example}

\section{Records}

New types can be constructed using the \code{define-record-type} macro
 from the \code{define-record-types} structure
The general syntax is:
\begin{example}
(define-record-type \cvar{tag} \cvar{type-name}
  (\cvar{constructor-name} \cvar{field-tag} \ldots)
  \cvar{predicate-name}
  (\cvar{field-tag} \cvar{accessor-name} [\cvar{modifier-name}])
  \ldots)
\end{example}
This makes the following definitions:
\begin{protos}
\constproto{\cvar{type-name}}{type}
\proto{\cvar{constructor-name}}{ field-init \ldots}{type-name}
\proto{\cvar{predicate-name}}{ value}{boolean}
\proto{\cvar{accessor-name}}{ type-name}{value}
\protonoresult{\cvar{modifier-name}}{ type-name value}
\end{protos}
\noindent
\cvar{Type-name} is the record type itself, and can be used to
 specify a print method (see below).
\cvar{Constructor-name} is a constructor that accepts values
 for the fields whose tags are specified.
\cvar{Predicate-name} to a predicate that can returns \code{\#t} for
 elements of the type and \code{\#f} for everything else.
The \cvar{accessor-name}s retrieve the values of fields,
 and the \cvar{modifier-name}'s update them.
The \cvar{tag} is used in printing instances of the record type and
 the field tags are used in the inspector and to match
 constructor arguments with fields.

\begin{protos}
\protonoresult{define-record-discloser}{ type discloser}
\end{protos}
\noindent
\code{Define-record-discloser} determines how
 records of type \cvar{type} are printed.
\cvar{Discloser} should be procedure which takes a single
 record of type \cvar{type} and returns a list whose car is
 a symbol.
The record will be printed as the value returned by \cvar{discloser}
 with curly braces used instead of the usual parenthesis.

For example
\begin{example}
(define-record-type pare :pare
  (kons x y)
  pare?
  (x kar set-kar!)
  (y kdr))
\end{example}
 defines \code{kons} to be a constructor, \code{kar} and \code{kdr} to be
 accessors, \code{set-kar!} to be a modifier, and \code{pare?} to be a predicate
 for a new type of object.
The type itself is named \code{:pare}.
\code{Pare} is a tag used in printing the new objects.

By default, the new objects print as \code{\#{Pare}}.
The print method can be modified using DEFINE-RECORD-DISCLOSER:
\begin{example}
(define-record-discloser :pare
  (lambda (p) `(pare ,(kar p) ,(kdr p))))
\end{example}
 will cause the result of \code{(kons 1 2)} to print as
 \code{\#\{pare 1 2\}}.

\section{Finite record types}

The structure \code{finite-types} has
 two macros for defining `finite' record types.
These are record types for which there are a fixed number of instances,
 which are created when the record type is defined.
The syntax for the defining a finite type is:
\begin{example}
(define-finite-type \cvar{tag} \cvar{type-name}
  (\cvar{field-tag} \ldots)
  \cvar{predicate-name}
  \cvar{vector-of-elements-name}
  \cvar{name-accessor}
  \cvar{index-accessor}
  (\cvar{field-tag} \cvar{accessor-name} [\cvar{modifier-name}])
  \ldots
  ((\cvar{element-name} \cvar{field-value} \ldots)
   \ldots))
\end{example}
This differs from \code{define-record-type} in the following ways:
\begin{itemize}
\item No name is specified for the constructor, but the field arguments
 to the constructor are listed.
\item The \cvar{vector-of-elements-name} is added; it will be bound
 to a vector containing all of the elements of the type.
These are constructed by applying the (unnamed) constructor to the
 initial field values at the end of the form.
\item There are names for accessors for two required fields, name
 and index.
These fields are not settable, and are not to be included
 in the argument list for the constructor.
\item The form ends with the names and the initial field values for
 the elements of the type.
The name must be first.
The remaining values must match the \cvar{field-tag}s in the constructor's
 argument list.
\item \cvar{Tag} is bound to a macro that maps \cvar{element-name}s to the
 the corresponding element of the vector.
The name lookup is done at macro-expansion time.
\end{itemize}

\begin{example}
(define-finite-type color :color
  (red green blue)
  color?
  colors
  color-name
  color-index
  (red   color-red)
  (green color-green)
  (blue  color-blue)
  ((white  255 255 255)
   (black    0   0   0)
   (yellow 255 255   0)
   (maroon 176  48  96)))

(color-name (vector-ref colors 0)) \evalsto white
(color-name (color black))         \evalsto black
(color-index (color yellow))       \evalsto 2
(color-red (color maroon))         \evalsto 176
\end{example}

Enumerated types are finite types whose only fields are the name
 and the index.
The syntax for defining an enumerated type is:
\begin{example}
(define-enumerated-type \cvar{tag} \cvar{type-name}
  \cvar{predicate-name}
  \cvar{vector-of-elements-name}
  \cvar{name-accessor}
  \cvar{index-accessor}
  (\cvar{element-name} \ldots))
\end{example}
In the absence of any additional fields, both the constructor argument
 list and the initial field values are not required.

The above example of a finite type can be pared down to the following
 enumerated type:
\begin{example}
(define-enumerated-type color :color
  color?
  colors
  color-name
  color-index
  (white black yellow maroon))

(color-name (vector-ref colors 0)) \evalsto white
(color-name (color black))         \evalsto black
(color-index (color yellow))       \evalsto 2
\end{example}

\section{Hash tables}

These are generic hash tables, and are in the structure \code{tables}.
Strictly speaking they are more maps than tables, as every table has a
 value for every possible key (for that type of table).
All but a finite number of those values are \code{\#f}.

\begin{protos}
\proto{make-table}{}{table}
\proto{make-symbol-table}{}{symbol-table}
\proto{make-string-table}{}{string-table}
\proto{make-integer-table}{}{integer-table}
\proto{make-table-maker}{ compare-proc hash-proc}{procedure}
\protonoresult{make-table-immutable!}{ table}
\end{protos}
\noindent
The first four functions listed make various kinds of tables.
\code{Make-table} returns a table whose keys may be symbols, integer,
 characters, booleans, or the empty list (these are also the values
 that may be used in \code{case} expressions).
As with \code{case}, comparison is done using \code{eqv?}.
The comparison procedures used in symbol, string, and integer tables are
 \code{eq?}, \code{string=?}, and \code{=}.

\code{Make-table-maker} takes two procedures as arguments and returns
 a nullary table-making procedure.
\cvar{Compare-proc} should be a two-argument equality predicate.
\cvar{Hash-proc} should be a one argument procedure that takes a key
 and returns a non-negative integer hash value.
If \code{(\cvar{compare-proc} \cvar{x} \cvar{y})} returns true,
 then \code{(= (\cvar{hash-proc} \cvar{x}) (\cvar{hash-proc} \cvar{y}))}
 must also return true.
For example, \code{make-integer-table} could be defined
 as \code{(make-table-maker = abs)}.

\code{Make-table-immutable!} prohibits future modification to its argument.

\begin{protos}
\proto{table?}{ value}{boolean}
\proto{table-ref}{ table key}{value or \code{\#f}}
\protonoresult{table-set!}{ table key value}
\protonoresult{table-walk}{ procedure table}
\end{protos}
\noindent
\code{Table?} is the predicate for tables.
\code{Table-ref} and \code{table-set!} access and modify the value of \cvar{key}
 in \cvar{table}.
\code{Table-walk} applies \cvar{procedure}, which must accept two arguments,
 to every associated key and non-\code{\#f} value in \code{table}.

\begin{protos}
\proto{default-hash-function}{ value}{integer}
\proto{string-hash}{ string}{integer}
\end{protos}
\noindent
\code{default-hash-function} is the hash function used in the tables
 returned by \code{make-table}, and \code{string-hash} it the one used
 by \code{make-string-table}.

%\W \chapter*{Index}
%\W \htmlprintindex
%\T \input{doc.ind}

\end{document}




\chapter{Access to POSIX}

This chapter describes Scheme~48's interface to the POSIX C calls
 \cite{POSIX}.
Scheme versions of most of the functions in POSIX are provided.
Both the interface and implementation are new and are likely to
 change in future releases.
Section~\ref{function-correspondence} lists which Scheme functions
 call which C functions.

Scheme~48's POSIX interface will likely change significantly in the
 future.
The implementation is new and may have significant bugs.

The POSIX bindings are available in several structures:

\begin{center}
\begin{tabular}{ll}
 \code{posix-processes} & fork, exec, and friends \\
 \code{posix-process-data} & information about processes \\
 \code{posix-files} & files and directories \\
 \code{posix-i/o} & operations on ports \\
 \code{posix-time} & time functions \\
 \code{posix-users} & users and groups \\
 \code{posix-regexps} & regular expression matching \\
 \code{posix-syslog} & POSIX logging facility \\
 \code{posix-errnos} & POSIX error codes\\
 \code{posix} & all of the above
\end{tabular}
\end{center}

Scheme~48's POSIX interface differs from
 Scsh's \cite{Shivers:Scsh-manual,Shivers:Scsh96} in several ways.
The interface here lacks Scsh's high-level constructs and utilities,
 such as the process notation, \code{awk} procedure, and parsing
 utilities.
Scheme~48 uses distinct types for some values that Scsh leaves
 as symbols or unboxed integers; these include file types, file modes,
 and user and group ids.
Many of the names and other interface details are different, as well.

\section{Process primitives}

The procedures described in this section control the creation of processes
 and the execution of programs.
They are in the structures \code{posix-process} and \code{posix}.

\subsection{Process creation and termination}
\label{processes}

\begin{protos}
\proto{fork}{}{process-id or {\tt \#f}}
\protonoresult{fork-and-forget}{ thunk}
% Handler versions?
\end{protos}
\noindent
\code{Fork} creates a new child process and returns the child's process-id in
 the parent and \code{\#f} in the child.
\code{Fork-and-forget} calls \cvar{thunk} in a new process; no process-id
 is returned.
\code{Fork-and-forget} uses an intermediate process to avoid creating
 a zombie process.

\begin{protos}
\proto{process-id?}{ x}{boolean}
\proto{process-id=?}{ process-id0 process-id1}{boolean}
\proto{process-id->integer}{ process-id}{integer}
\proto{integer->process-id}{ integer}{process-id}
\end{protos}
\noindent
%A \cvar{process-id} is a  around the actual integer id.
\code{Process-id?} is a predicate for process-ids,
\code{process-id=?} compares two to see if they are the same,
 and \code{process-id-uid} returns the actual Unix id.
\code{Process-id->integer} and \code{integer->process-id}
 convert process ids to and from integers.

\begin{protos}
\proto{process-id-exit-status}{ process-id}{integer or {\tt \#f}}
\proto{process-id-terminating-signal}{ process-id}{signal or {\tt \#f}}
\protonoresult{wait-for-child-process}{ process-id}
\end{protos}
\noindent
If a process terminates normally
 \code{process-id-exit-status} will return its exit status.
If the process is still running or was terminated by a signal then
 \code{process-id-exit-status} will return \code{\#f}.
Similarly, if a child process was terminated by a signal
 \code{process-id-terminating-signal} will return that signal and
 will return \code{\#f} if the process is still running or terminated
 normally.
\code{Wait-for-child-process} blocks until the child process terminates.
Scheme~48 may reap child processes before the user requests their
 exit status, but it does not always do so.

\begin{protos}
\protonoresult{exit}{ status}
\end{protos}
\noindent
Terminates the current process with the integer \cvar{status}
 as its exit status.

\subsection{{\tt Exec}}

\begin{protos}
\protonoresult{exec}{ program-name arg0 \ldots}
\protonoresult{exec-with-environment}{ program-name env arg0 \ldots}
\protonoresult{exec-file}{ filename arg0 \ldots}
\protonoresult{exec-file-with-environment}{ filename env arg0 \ldots}
\protonoresult{exec-with-alias}{ name lookup? maybe-env arguments}
\end{protos}
\noindent
All of these replace the current program with a new one.
They differ in how the new program is found, what its environment is,
 and what arguments it is passed.
\code{Exec} and \code{exec-with-environment}
 look up the new program in the search path,
 while \code{exec-file} and \code{exec-file-with-environment}
 execute a particular file.
The environment is either inherited from the current process
 (\code{exec} and \code{exec-file}) or given as an argument
 (\code{\ldots{}-with-environment}).
\cvar{Program-name} and \cvar{filename} and any \cvar{arg$_i$} should
be \var{os-string-thing} arguments (see section~\ref{os-strings}.
 \cvar{Env} should be a list of \var{os-string-thing} arguments of the form
 \code{"\cvar{name}=\cvar{value}"}.
The first four procedures add their first argument, \cvar{program-name} or
 \cvar{filename}, before the \cvar{arg0 \ldots} arguments.

\code{Exec-with-alias} is an omnibus procedure that subsumes the other
 four.
\cvar{Name} is looked up in the search path if \cvar{lookup?} is true
 and is used as a filename otherwise.
\cvar{Maybe-env} is either a list of \var{os-string-thing}s for the environment of the
 new program or \code{\#f} in which case the new program inherits its
 environment from the current one.
\cvar{Arguments} should be a list of \var{os-string-thing}s; unlike with the other four
 procedures, \cvar{name} is not added to this list (hence \code{-with-alias}).

\section{Signals}

There are two varieties of signals available, {\em named} and {\em anonymous}.
A named signal is one for which we have a symbolic name, such as \code{kill}
 or \code{pipe}.
Anonymous signals, for which we only have the current operating system's
 signal number, have no meaning in other operating systems.
Named signals preserve their meaning in image files.
Not all named signals are available from all OS's and
 there may be multiple names for a single OS signal number.

\begin{protos}
\syntaxproto{signal}{ \cvar{signal-name}}{signal}
\proto{name->signal}{ symbol}{signal or {\tt \#f}}
\proto{integer->signal}{ integer}{signal}
\proto{signal?}{ x}{boolean}
\proto{signal-name}{ signal}{symbol or {\tt \#f}}
\proto{signal-os-number}{ signal}{integer}
\proto{signal=?}{ signal0 signal1}{boolean}
\end{protos}
\noindent
The syntax \code{signal} returns a (named) signal associated with
\cvar{signal-name}.
\code{Name->signal} returns a (named) signal or \code{\#f} if the
 the signal \cvar{name} is not supported by the operating system.
The signal returned by \code{integer->signal} is a named signal if
 \cvar{integer} corresponds to a named signal in the current operating
 system; otherwise it returns an anonymous signal.
\code{Signal-name} returns a symbol if \cvar{signal} is named and
 \code{\#f} if it is anonymous.
\code{Signal=?} returns \code{\#t} if \cvar{signal0} and \cvar{signal1}
 have the same operating system number and \code{\#f} if they do not.

\subsection{POSIX signals}
The following lists the names of the POSIX signals.
\begin{center}
\begin{tabular}{ll}
\code{abrt} & abort - abnormal termination (as by abort()) \\
\code{alrm} & alarm - timeout signal (as by alarm()) \\
\code{fpe } & floating point exception \\
\code{hup } & hangup - hangup on controlling terminal or death of
  controlling process \\
\code{ill } & illegal instruction \\
\code{int } & interrupt - interaction attention \\
\code{kill} & kill - termination signal, cannot be caught or ignored \\
\code{pipe} & pipe - write on a pipe with no readers \\
\code{quit} & quit - interaction termination \\
\code{segv} & segmentation violation - invalid memory reference \\
\code{term} & termination - termination signal \\
\code{usr1} & user1 - for use by applications \\
\code{usr2} & user2 - for use by applications \\
\code{chld} & child - child process stopped or terminated \\
\code{cont} & continue - continue if stopped \\
\code{stop} & stop - cannot be caught or ignored \\
\code{tstp} & interactive stop \\
\code{ttin} & read from control terminal attempted by background process \\
\code{ttou} & write to control terminal attempted by background process \\
\code{bus } & bus error - access to undefined portion of memory \\
\end{tabular}
\end{center}

\subsection{Other signals}
The following lists the names of the non-POSIX signals that the system is
 currently aware of.
\begin{center}
\texonly
\setlongtables
\begin{longtable}{ll}
\endtexonly
\htmlonly\begin{tabular}{ll}\endhtmlonly
\code{trap  } & trace or breakpoint trap \\
\code{iot   } & IOT trap - a synonym for ABRT \\
\code{emt   } & \\
\code{sys   } & bad argument to routine (SVID) \\
\code{stkflt} & stack fault on coprocessor \\
\code{urg   } & urgent condition on socket      (4.2 BSD) \\
\code{io    } & I/O now possible                (4.2 BSD) \\
\code{poll  } & A synonym for SIGIO             (System V) \\
\code{cld   } & A synonym for SIGCHLD \\
\code{xcpu  } & CPU time limit exceeded (4.2 BSD) \\
\code{xfsz  } & File size limit exceeded        (4.2 BSD) \\
\code{vtalrm} & Virtual alarm clock             (4.2 BSD) \\
\code{prof  } & Profile alarm clock \\
\code{pwr   } & Power failure                   (System V) \\
\code{info  } & A synonym for SIGPWR \\
\code{lost  } & File lock lost \\
\code{winch } & Window resize signal            (4.3 BSD, Sun) \\
\code{unused} & Unused signal \\
\texonly\end{longtable}\endtexonly
\htmlonly\end{tabular}\endhtmlonly
\end{center}

\subsection{Sending signals}

\begin{protos}
\protonoresult{signal-process}{ process-id signal}
\end{protos}
\noindent
Send \cvar{signal} to the process corresponding to \cvar{process-id}.

\subsection{Receiving signals}

Signals received by the Scheme process can be obtained via one or more
 signal-queues.
Each signal queue has a list of monitored signals and a queue of
 received signals that have yet to be read from the signal-queue.
When the Scheme process receives a signal that signal is added to the
 received-signal queues of all signal-queues which are currently monitoring
 that particular signal.

\begin{protos}
\proto{make-signal-queue}{ signals}{signal-queue}
\proto{signal-queue?}{ x}{boolean}
\proto{signal-queue-monitored-signals}{ signal-queue}{list of signals}
\protonoindex{dequeue-signal!}{ signal-queue}{signal}\mainschindex{dequeue-signal"!}
\protonoindex{maybe-dequeue-signal!}{ signal-queue}{signal or {\tt \#f}}\mainschindex{maybe-dequeue-signal"!}
\end{protos}
\noindent
\code{Make-signal-queue} returns a new signal-queue that will monitor
 the signals in the list \cvar{signals}.
\code{Signal-queue?} is a predicate for signal queues.
\code{Signal-queue-monitored-signals} returns a list of the signals
 currently monitored by \cvar{signal-queue}.
\code{Dequeue-signal!} and \code{maybe-dequeue-signal} both return
 the next received-but-unread signal from \cvar{signal-queue}.
If \cvar{signal-queue}'s queue of signals is empty \code{dequeue-signal!}
 blocks until an appropriate signal is received.
\code{Maybe-dequeue-signal!} does not block; it returns \code{\#f} instead.

There is a bug in the current system that causes an erroneous deadlock
 error if threads are blocked waiting for signals and no other threads
 are available to run.
A work around is to create a thread that sleeps for a long time, which
 prevents any deadlock errors (including real ones):
\begin{example}
> ,open threads
> (spawn (lambda ()
           ; Sleep for a year
           (sleep (* 1000 60 60 24 365))))
\end{example}

\begin{protos}
\protonoresultnoindex{add-signal-queue-signal!}{ signal-queue signal}\mainschindex{add-signal-queue-signal"!}
\protonoresultnoindex{remove-signal-queue-signal!}{ signal-queue signal}\mainschindex{remove-signal-queue-signal"!}
\end{protos}
\noindent
These two procedures can be used to add or remove signals from a
 signal-queue's list of monitored signals.
When a signal is removed from a signal-queue's list of monitored signals
 any occurances of the signal are removed from that signal-queue's pending
 signals.
In other words, \code{dequeue-signal!} and \code{maybe-dequeue-signal!}
 will only return signals that are currently on the signal-queue's list
 of signals.

\section{Process environment}

These are in structures \code{posix-process-data} and \code{posix}.

\subsection{Process identification}

\begin{protos}
\proto{get-process-id}{}{ process-id}
\proto{get-parent-process-id}{}{ process-id}
\end{protos}
\noindent
These return the process ids of the current process and its parent.
See section~\ref{processes} for operations on process ids.

\begin{protos}
\proto{get-user-id}{}{ user-id}
\proto{get-effective-user-id}{}{ user-id}
\protonoresultnoindex{set-user-id!}{ user-id}\mainschindex{set-user-id"!}
\end{protos}

\begin{protos}
\proto{get-group-id}{}{ group-id}
\proto{get-effective-group-id}{}{ group-id}
\protonoresultnoindex{set-group-id!}{ group-id}\mainschindex{set-group-id"!}
\end{protos}
\noindent
Every process has both the original and effective user id and group id.
The effective values may be set, but not the original ones.

\begin{protos}
\proto{get-groups}{}{ group-ids}
\proto{get-login-name}{}{ os-string}
\end{protos}
\noindent
\code{Get-groups} returns a list of the supplementary groups of the
 current process.
\code{Get-login-name} returns a user name for the current process.

% 4.5 Get Process Times

% 4.6 Environment Variables

\subsection{Environment variables}

\begin{protos}
\proto{lookup-environment-variable}{ os-string-thing}{os-string or {\tt \#f}}
\proto{environment-alist}{}{alist}
\end{protos}
\noindent
\code{Lookup-environment-variable} looks up its argument in the
 environment list and returns the corresponding value or \code{\#f}
 if there is none.
\code{Environment-alist} returns the entire environment as a list of
 \code{(\cvar{name-os-string} . \cvar{value-os-string})} pairs.

\section{Users and groups}

\cvar{User-id}\/s and \cvar{group-id}\/s are boxed integers representing
 Unix users and groups.
The procedures in this section are in structures \code{posix-users} and
 \code{posix}.

\begin{protos}
\proto{user-id?}{ x}{boolean}
\proto{user-id=?}{ user-id0 user-id1}{boolean}
\proto{user-id->integer}{ user-id}{integer}
\proto{integer->user-id}{ integer}{user-id}
\end{protos}

\begin{protos}
\proto{group-id?}{ x}{boolean}
\proto{group-id=?}{ group-id0 group-id1}{boolean}
\proto{group-id->integer}{ group-id}{integer}
\proto{integer->group-id}{ integer}{group-id}
\end{protos}
\noindent
User-ids and group-ids have their own
 own predicates and comparison, boxing, and unboxing functions.

\begin{protos}
\proto{user-id->user-info}{ user-id}{user-info}
\proto{name->user-info}{ os-string}{user-info}
\end{protos}
\noindent
These return the user info for a user identified by user-id or name.

\begin{protos}
\proto{user-info?}{ x}{ boolean}
\proto{user-info-name}{ user-info}{ os-string}
\proto{user-info-id}{ user-info}{ user-id}
\proto{user-info-group}{ user-info}{ group-id}
\proto{user-info-home-directory}{ user-info}{ os-string}
\proto{user-info-shell}{ user-info}{ os-string}
\end{protos}
\noindent
A \code{user-info} contains information about a user.
Available are the user's name, id, group, home directory, and shell.

\begin{protos}
\proto{group-id->group-info}{ group-id}{group-info}
\proto{name->group-info}{ os-string}{group-info}
\end{protos}
\noindent
These return the group info for a group identified by group-id or name.

\begin{protos}
\proto{group-info?}{ x}{ boolean}
\proto{group-info-name}{ group-info}{ os-string}
\proto{group-info-id}{ group-info}{ group-id}
\proto{group-info-members}{ group-info}{ user-ids}
\end{protos}
\noindent
A \code{group-info} contains information about a group.
Available are the group's name, id, and a list of members.

\section{OS and machine identification}

These procedures return strings that are supposed to identify the current
 OS and machine.
The POSIX standard does not indicate the format of the strings.
The procedures are in structures \code{posix-platform-names} and \code{posix}.

\begin{protos}
\proto{os-name}{}{string}
\proto{os-node-name}{}{string}
\proto{os-release-name}{}{string}
\proto{os-version-name}{}{string}
\proto{machine-name}{}{string}
\end{protos}

% 4.8 Configurable System Variables

\section{Files and directories}

These procedures are in structures \code{posix-files} and \code{posix}.

\subsection{Directory streams}

Directory streams are like input ports, with each read operation
 returning the next name in the directory.

\begin{protos}
\proto{open-directory-stream}{ name}{directory}
\proto{directory-stream?}{ x}{boolean}
\proto{read-directory-stream}{ directory}{name or {\tt \#f}}
\protonoresult{close-directory-stream}{ directory}
\end{protos}
\noindent
\code{Open-directory-stream} opens a new directory stream.
\code{Directory-stream?} is a predicate that recognizes directory streams.
\code{Read-directory-stream} returns the next name in the directory or
 \code{\#f} if all names have been read.
\code{Close-directory-stream} closes a directory stream.

\begin{protos}
\proto{list-directory}{ name}{list of os-strings}
\end{protos}
\noindent
This is the obvious utility; it returns a list of the names in directory
 \cvar{name}.

% 5.2 Working Directory
\subsection{Working directory}

\begin{protos}
\proto{working-directory}{}{os-string}
\protonoresultnoindex{set-working-directory!}{ os-string-thing}\mainschindex{set-working-directory"!}
\end{protos}
\noindent
These return and set the working directory.

% 5.3 File Creation
\subsection{File creation and removal}

\begin{protos}
\proto{open-file}{ path file-options}{port}
\proto{open-file}{ path file-options file-mode}{port}
\end{protos}
\noindent
\code{Open-file} opens a port to the file named by \cvar{path}, which
must be a \var{os-string-thing} argument.
The \cvar{file-options} argument determines various aspects of the
 returned port.
The optional \cvar{file-mode} argument is used only if the file to be opened
 does not already exist.
The returned port is an input port if \cvar{file-options} includes
 \code{read-only}; otherwise it returns an output port.
\code{Dup-switching-mode} can be used to open an input port for
 output ports opened with the \code{read/write} option.

\begin{protos}
\syntaxproto{file-options}{ \cvar{file-option-name} \ldots}{file-options}
\proto{file-options-on?}{ file-options file-options}{boolean}
\proto{file-options-union}{ file-options file-options}{file-options}
\end{protos}
\noindent
The syntax \code{file-options} returns a file-option with the
 indicated options set.
\code{File-options-on?} returns true if its first argument includes all of
 the options listed in the second argument.  \code{File-options-union}
returns a file-options argument containing exactly all of the options
listed in either argument.
The following file options may be used with \code{open-file}.

\begin{center}
\begin{tabular}{llp{2.8in}}
 & \code{create} & create file if it does not already exist; a file-mode argument
 is required with this option \\
 & \code{exclusive} & an error will be raised if this option and \code{create}
 are both set and the file already exists \\
 & \code{no-controlling-tty} & if \cvar{path} is a terminal device this option
 causes the terminal to not become the controlling terminal of the process\\
 & \code{truncate} & file is truncated \\
 & \code{append} & writes are appended to existing contents \\
 & \code{nonblocking} & read and write operations do not block \\
 & \code{read-only} & port may not be written \\
 & \code{read-write} & file descriptor may be read or written \\
 & \code{write-only} & port may not be read
\end{tabular}
\end{center}
\noindent Only one of the last three options may be used.  If
\code{read-write} is specified, an output port is returned.

For example
\begin{example}
(open-file "some-file.txt"
           (file-options create write-only)
           (file-mode read owner-write))
\end{example}
 returns an output port that writes to a newly-created file that can be
 read by anyone and written only by the owner.
Once the file exists,
\begin{example}
(open-file "some-file.txt"
           (file-options append write-only))
\end{example}
will open an output port that appends to the file.

The \code{append} and \code{nonblocking} options and the read/write nature of
 the port can be read using \code{i/o-flags}.
The \code{append} and \code{nonblocking} options can be set
 using \code{set-i/o-flags!}.

To keep port operations from blocking the Scheme~48 process, output
 ports are set to be nonblocking at the time of creation (input ports
 are managed using \code{select()}).
You can use \code{set-i/o-flags!} to make an output port blocking, for
 example just before a fork, but care should be exercised.
The Scheme~48 runtime code may get confused if an I/O operation blocks.

\begin{protos}
\protonoresultnoindex{set-file-creation-mask!}{ file-mode}\mainschindex{set-file-creation-mask"!}
\end{protos}
\noindent
Sets the file creation mask to be \cvar{file-mode}.
Bits set in \cvar{file-mode} are cleared in the modes of any files or
 directories created by the current process.

\begin{protos}
\protonoresult{link}{ existing new}
\end{protos}
\noindent
Both \var{existing} and \var{new} must be \var{os-string-thing} arguments.
\code{Link} makes path \cvar{new} be a new link to the file
 pointed to by path \cvar{existing}.
The two paths must be in the same file system.

% 5.4 Special File Creation

\begin{protos}
\protonoresult{make-directory}{ path file-mode}
\protonoresult{make-fifo}{ path file-mode}
\end{protos}
\noindent
These two procedures make new directories and fifo files.  In both
cases, \var{path} must be a \var{os-string-thing} argument.

% 5.5 File Removal

\begin{protos}
\protonoresult{unlink}{ path}
\protonoresult{remove-directory}{ path}
\protonoresult{rename}{ old-path new-path}
\end{protos}
\noindent
\var{Path}, \var{old-path} and \var{new-path} must all be
\var{os-string-thing} arguments.
\code{Unlink} removes the link indicated by \cvar{path}.
\code{Remove-directory} removes the indicated (empty) directory.
\code{Rename} moves the file pointed to by \cvar{old-path} to the
 location pointed to by \cvar{new-path} (the two paths must be in
 the same file system).
Any other links to the file remain unchanged.

% 5.6.3 (out of POSIX order)

\begin{protos}
\proto{accessible?}{ path access-mode . more-modes}{boolean}
\syntaxproto{access-mode}{ \cvar{mode-name}}{access-mode}
\end{protos}
\noindent
\code{Accessible?} returns true if \cvar{path} (which must be a
\var{os-string-thing} argument) is a file that
 can be accessed in the listed mode.
If more than one mode is specified \code{accessible?} returns true
 if all of the specified modes are permitted.
The \cvar{mode-name}\/s are: \code{read}, \code{write}, \code{execute},
 \code{exists}.

% 5.6 File Characteristics
\subsection{File information}

\begin{protos}
\proto{get-file-info}{ path}{file-info}
\proto{get-file/link-info}{ name}{file-info}
\proto{get-port-info}{ fd-port}{file-info}
\end{protos}
\noindent
\code{Get-file-info} and \code{get-file/link-info} both return
 a file info record for the file named by \var{path}, which must be a
 \var{os-string-thing} argument.
\code{Get-file-info} follows symbolic links while \code{get-file/link-info}
 does not.
\code{Get-port-info} returns a file info record for the file
 which \cvar{port} reads from or writes to.
An error is raised if \cvar{fd-port} does not read from or write to a
 file descriptor.

\begin{protos}
\proto{file-info?}{ x}{boolean}
\proto{file-info-name}{ file-info}{os-string}
\end{protos}
\noindent
\code{File-info?} is a predicate for file-info records.
\code{File-info-name} is the name which was used to get \code{file-info},
 either as passed to \code{get-file-info} or \code{get-file/link-info},
 or used to open the port passed to \code{get-port-info}.

\begin{protos}
\proto{file-info-type}{ file-info}{file-type}
\proto{file-type?}{ x}{boolean}
\proto{file-type-name}{ file-type}{symbol}
\syntaxproto{file-type}{ \cvar{type}}{file-type}
\end{protos}
\noindent
%I need to add the rest - the procedure version of file-type, enum-case(?).
\code{File-info-type} returns the type of the file, as a file-type object
File types may be compared using \code{eq?}.
The valid file types are:

\begin{center}
\begin{tabular}{l}
 \code{regular} \\
 \code{directory} \\
 \code{character-device} \\
 \code{block-device} \\
 \code{fifo} \\
 \code{symbolic-link} \\
 \code{socket} \\
 \code{other}
\end{tabular}
\end{center}

\noindent
\code{Symbolic-link} and \code{socket} are not required by POSIX.

\begin{protos}
\proto{file-info-device}{ file-info}{integer}
\proto{file-info-inode}{ file-info}{integer}
\end{protos}
\noindent
The device and inode numbers uniquely determine a file.

\begin{protos}
\proto{file-info-link-count}{ file-info}{integer}
\proto{file-info-size}{ file-info}{integer}
\end{protos}
\noindent
These return the number of links to a file and the file size in bytes.
The size is only meaningful for regular files.

\begin{protos}
\proto{file-info-owner}{ file-info}{user-id}
\proto{file-info-group}{ file-info}{group-id}
\proto{file-info-mode}{ file-info}{file-mode}
\end{protos}
\noindent
These return the owner, group, and access mode of a file.

\begin{protos}
\proto{file-info-last-access}{ file-info}{time}
\proto{file-info-last-modification}{ file-info}{time}
\proto{file-info-last-info-change}{ file-info}{time}
\end{protos}
\noindent
These return the time the file was last read, modified, or had its
 status modified

% Modes
\subsection{File modes}

A file mode is a boxed integer representing a file protection mask.

\begin{protos}
\syntaxproto{file-mode}{ permission-name \ldots}{file-mode}
\proto{file-mode?}{ x}{boolean}
\proto{file-mode+}{ file-mode \ldots}{file-mode}
\proto{file-mode-}{ file-mode0 file-mode1}{file-mode}
\end{protos}
\noindent
\code{File-mode} is syntax for creating file modes.
The mode-names are listed below.
\code{File-mode?} is a predicate for file modes.
\code{File-mode+} returns a mode that contains all of permissions of
 its arguments.
\code{File-mode-} returns a mode that has all of the permissions of
 \cvar{file-mode0} that are not in \cvar{file-mode1}.

\begin{protos}
\proto{file-mode=?}{ file-mode0 file-mode1}{boolean}
\proto{file-mode<=?}{ file-mode0 file-mode1}{boolean}
\proto{file-mode>=?}{ file-mode0 file-mode1}{boolean}
\end{protos}
\noindent
\code{File-mode=?} returns true if the two modes are exactly the same.
\code{File-mode<=?} returns true if \cvar{file-mode0} has a subset
 of the permissions of \cvar{file-mode1}.
\code{File-mode>=?} is \code{file-mode<=?} with the arguments reversed.

\begin{protos}
\proto{file-mode->integer}{ file-mode}{integer}
\proto{integer->file-mode}{ integer}{file-mode}
\end{protos}
\noindent
\code{Integer->file-mode} and \code{file-mode->integer} translate file modes
 to and from the classic Unix file mode masks.
These may not be the masks used by the underlying OS.

\begin{center}
\begin{tabular}{lll}
Permission name & Bit mask \\
\code{set-uid} & \code{\#o4000} & set user id when executing \\
\code{set-gid} & \code{\#o2000} & set group id when executing \\
\code{owner-read} & \code{\#o0400} & read by owner \\
\code{owner-write} & \code{\#o0200} & write by owner \\
\code{owner-exec} & \code{\#o0100} & execute (or search) by owner \\
\code{group-read} & \code{\#o0040} & read by group \\
\code{group-write} & \code{\#o0020} & write by group \\
\code{group-exec} & \code{\#o0010} & execute (or search) by group \\
\code{other-read} & \code{\#o0004} & read by others \\
\code{other-write} & \code{\#o0002} & write by others \\
\code{other-exec} & \code{\#o0001} & execute (or search) by others \\
\end{tabular}
\end{center}

\begin{center}
\begin{tabular}{lll}
\multicolumn{3}{l}{Names for sets of permissions} \\
\code{owner} & \code{\#o0700} & read, write, and execute by owner \\
\code{group} & \code{\#o0070} & read, write, and execute by group \\
\code{other} & \code{\#o0007} & read, write, and execute by others \\
\code{read} & \code{\#o0444} & read by anyone \\
\code{write} & \code{\#o0222} & write by anyone \\
\code{exec} & \code{\#o0111} & execute by anyone \\
\code{all} & \code{\#o0777} & anything by anyone
\end{tabular}
\end{center}

\subsection{Symbolic links}

\begin{protos}
\protonoresult{create-symbolic-link}{ path1 path2}
\end{protos}
%
This creates a symbolic link at \var{path2} that contains \var{path1}.
\var{Path1} and \var{path2} must be \var{os-string-thing} arguments.

\begin{protos}
\proto{read-symbolic-link}{ path}{ os-string}
\end{protos}
%
This returns contents of the symbolic link at \var{path}.
\var{Path} must be an \var{os-string-thing} argument.

%----------------
% Time - seconds since the epoch.
\section{Time and Date}

These procedures are in structures \code{posix-time} and \code{posix}.

\subsection{Time}

\begin{protos}
\proto{make-time}{ integer}{time}
\proto{current-time}{}{time}
\proto{time?}{ x}{boolean}
\proto{time-seconds}{ time}{integer}
\end{protos}
\noindent
A \code{time} record contains an integer that represents time as
 the number of second since the Unix epoch (00:00:00 GMT, January 1, 1970).
\code{Make-time} and \code{current-time} return \code{time}\/s, with
 \code{make-time}'s using its argument while \code{current-time}'s has
 the current time.
\code{Time?} is a predicate that recognizes \code{time}\/s and
 \code{time-seconds} returns the number of seconds \cvar{time} represents.

\begin{protos}
\proto{time=?}{ time time}{boolean}
\proto{time<?}{ time time}{boolean}
\proto{time<=?}{ time time}{boolean}
\proto{time>?}{ time time}{boolean}
\proto{time>=?}{ time time}{boolean}
\end{protos}
\noindent
These perform various comparison operations on the \code{time}\/s.

\begin{protos}
\proto{time->string}{ time}{string}
\end{protos}
\noindent
\code{Time->string} returns a string representation of \cvar{time} in the
locale's version of the following form.
\begin{example}
"Wed Jun 30 21:49:08 1993
"
\end{example}

\subsection{Date}

A date is a time specification relative to a specific but implicit
time zone, broken out into the familar
year-month-day-hour-minute-second format.

\begin{protos}
\proto{make-date}{ second minute hour month-day month year week-day year-day dst}{date}
\proto{date?}{ x}{boolean}
\proto{date-second}{ date}{second}
\proto{date-minute}{ date}{minute}
\proto{date-hour}{ date}{hour}
\proto{date-month-day}{ date}{month-day}
\proto{date-month}{ date}{month}
\proto{date-year}{ date}{year}
\proto{date-week-day}{ date}{week-day}
\proto{date-year-day}{ date}{year-day}
\proto{date-date-dst}{ date}{dst}
\end{protos}
%
These are the constructor, predicate and corresponding accessors for
date objects.  The meaning of the various field types are as follows:

\begin{center}
  \begin{tabular}{lll}
    \cvar{seconds} & seconds & (0--60) \\
    \cvar{minutes} & minutes & (0--59) \\
    \cvar{hour} & hours & (0--23) \\
    \cvar{month-day} & day of month & (1--31) \\
    \cvar{mon}     & month of year & (0--11) \\
    \cvar{year}    & year & since 1900\\
    \cvar{week-day}    & day of week & (Sunday = 0) \\
    \cvar{year-day}    & day of year & (0--365) \\
    \cvar{dst}   & is summer time in effect? & \code{\#t}, \code{\#f} or unspecific
  \end{tabular}
\end{center}

\begin{protos}
\proto{date->string}{ date}{string}
\end{protos}
%
returns a string representation of \cvar{date} in the
locale's version of the following form:
\begin{example}
"Wed Jun 30 21:49:08 1993
"
\end{example}

\begin{protos}
\proto{time->utc-date}{ time}{date}
\proto{time->local-date}{ time}{date}
\end{protos}
%
These convert a time object into a date object; the first does this
relative to the UTC time zone, the second relative to the current
timezone setting.

\begin{protos}
\proto{date->time}{ date}{time}
\end{protos}
%
This converts a date object into a time object relative to the current
timezone setting.

\begin{protos}
\proto{format-date}{ string date}{string}
\end{protos}
%
This formats a date into a string, according to the format
specification in the first argument.  The format specification is
according to the specification of the C \code{strftime} function:

\noindent
\begin{longtable}{lp{0.9\textwidth}}
\code{\%a} & is replaced by the locale's abbreviated weekday name.\\
\code{\%A} & is replaced by the locale's full weekday name.\\
\code{\%b} & is replaced by the locale's abbreviated month name.\\
\code{\%B} & is replaced by the locale's full month name.\\
\code{\%c} & is replaced by the locale's appropriate date and time representation.\\
\code{\%C} & is replaced by the year divided by 100 and truncated to an integer, as a decimal
number (00--99).\\
\code{\%d} & is replaced by the day of the month as a decimal number (01--31).\\
\code{\%D} & is equivalent to ``\verb|%m/%d/%y|''.\\
\code{\%e} & is replaced by the day of the month as a decimal number (1--31); a single digit
is preceded by a space.\\
\code{\%F} & is equivalent to ``\verb|%Y-%m-%d|'' (the ISO 8601 date format).\\
\code{\%g} & is replaced by the last 2 digits of the week-based year (see below) as a decimal
number (00--99).\\
\code{\%G} & is replaced by the week-based year (see below) as a decimal number (e.g.,
1997).\\
\code{\%h} & is equivalent to ``\verb|%b|''.\\
\code{\%H} & is replaced by the hour (24-hour clock) as a decimal number (00--23).\\
\code{\%I} & is replaced by the hour (12-hour clock) as a decimal number (01--12).\\
\code{\%j} & is replaced by the day of the year as a decimal number (001--366).\\
\code{\%m} & is replaced by the month as a decimal number (01--12).\\
\code{\%M} & is replaced by the minute as a decimal number (00--59).\\
\code{\%n} & is replaced by a new-line character.\\
\code{\%p} & is replaced by the locale's equivalent of the AM/PM designations associated
with a 12-hour clock.\\
\code{\%r} & is replaced by the locale's 12-hour clock time.\\
\code{\%R} & is equivalent to ``\verb|%H:%M|''.\\
\code{\%S} & is replaced by the second as a decimal number (00--60).\\
\code{\%t} & is replaced by a horizontal-tab character.\\
\code{\%T} & is equivalent to ``\verb|%H:%M:%S|'' (the ISO 8601 time format).\\
\code{\%u} & is replaced by the ISO 8601 weekday as a decimal number (1--7), where
Monday is 1.\\
\code{\%U} & is replaced by the week number of the year (the first Sunday as the first day of
week 1) as a decimal number (00--53).\\
\code{\%V} & is replaced by the ISO 8601 week number (see below) as a decimal number
(01--53).\\
\code{\%w} & is replaced by the weekday as a decimal number (0--6), where Sunday is 0.\\
\code{\%W} & is replaced by the week number of the year (the first Monday as the first day of
week 1) as a decimal number (00--53).\\
\code{\%x} & is replaced by the locale's appropriate date representation.\\
\code{\%X} & is replaced by the locale's appropriate time representation.\\
\code{\%y} & is replaced by the last 2 digits of the year as a decimal number (00--99).\\
\code{\%Y} & is replaced by the year as a decimal number (e.g., 1997).\\
\code{\%z} & is replaced by the offset from UTC in the ISO 8601 format ``\verb|-0430|'' (meaning
4 hours 30 minutes behind UTC, west of Greenwich), or by no characters if no
time zone is determinable.\\
\code{\%Z} & is replaced by the locale's time zone name or abbreviation, or by no characters if
no time zone is determinable.\\
\code{\%\%} & is replaced by \verb|%|.
\end{longtable}

\section{I/O}

These procedures are in structures \code{posix-i/o} and \code{posix}.

% Rest of 5.6

% int chmod(char *path, mode_t mode)
% int fchmod(int fd, mode_t mode)

% int chown(char *path, uid_t owner, gid_t group)
% int utime(char *path, struct utimbuf * times)

% int ftruncate(int fd, off_t length)

%----------------
% 5.7 Configurable Pathname Variables
%
% long pathconf(char *path, int name)
% long fpathconf(int fd, int name)

%----------------
% 6.1 Pipes

\begin{protos}
\proto{open-pipe}{}{input-port + output-port}
\end{protos}
\noindent
\code{Open-pipe} creates a new pipe and returns the two ends as an
 input port and an output port.

% 6.2 File descriptor manipulation.

A {\em file descriptor} port (or \cvar{fd-port})
 is a port that reads to or writes from an OS file descriptor.
Fd-ports are returned by \code{open-input-file}, \code{open-output-file},
 \code{open-file}, \code{open-pipe}, and other procedures.
% Richard says: add an example of a non-fd port once the extended ports
% are described in the manual.

\begin{protos}
\proto{fd-port?}{ port}{boolean}
\proto{port->fd}{ port}{integer or {\tt \#f}}
\end{protos}
\noindent
\code{Fd-port?} returns true if its argument is an fd-port.
\code{Port->fd} returns the file descriptor associated with
 or \code{\#f} if \cvar{port} is not an fd-port.

\begin{protos}
\protonoresult{remap-file-descriptors}{ fd-spec \ldots}
\end{protos}
\noindent
\code{Remap-file-descriptors} reassigns file descriptors to ports.
The \cvar{fd-specs} indicate which port is to be mapped to each
 file descriptor: the first gets file descriptor \code{0}, the second gets
 \code{1}, and so forth.
A \cvar{fd-spec} is either a port that reads from or writes to
 a file descriptor,
 or \code{\#f}, with \code{\#f} indicating that the corresponding file
 descriptor is not used.
Any open ports not listed are marked `close-on-exec'.
The same port may be moved to multiple new file descriptors.

For example,
\begin{example}
(remap-file-descriptors (current-output-port)
                        \#f
                        (current-input-port))
\end{example}
moves the current output port to file descriptor \code{0} and the
current input port to file descriptor \code{2}.

\begin{protos}
\proto{dup}{ fd-port}{fd-port}
\proto{dup-switching-mode}{ fd-port}{fd-port}
\proto{dup2}{ fd-port file-descriptor}{fd-port}
\end{protos}
\noindent
These change \cvar{fd-port}'s file descriptor and return a new port
 that uses \cvar{ports}'s old file descriptor.
\code{Dup} uses the lowest unused file descriptor and \code{dup2} uses the
 one provided.
\code{Dup-switching-mode} is the same as \code{dup} except that the returned
 port is an input port if the argument was an output port and vice
 versa.
If any existing port uses the file descriptor passed to \code{dup2}, that
 port is closed.

\begin{protos}
\protonoresult{close-all-but}{ port \ldots}
\end{protos}
\noindent
\code{Close-all-but} closes all file descriptors whose associated ports
 are not passed to it as arguments.

% 6.3 File Descriptor Reassignment
%
% int close(int fd)     % Use close-{input|output}-{port|channel}
%
% 6.4 Input and Output
%
% read() and write()    ; Already available in various forms.

% 6.5 Control Operations on Files

%; fcntl(fd, F_DUPFD, target_fd)            ; Use DUP instead.
%
%; Descriptor flags
%; fcntl(fd, F_GETFD)
%; fcntl(fd, F_SETFD, flags)
%;
%; The only POSIX flag is FD_CLOEXEC, so that's all we do.

\begin{protos}
\proto{close-on-exec?}{ port}{boolean}
\protonoresultnoindex{set-close-on-exec?!}{ port boolean}\mainschindex{set-close-on-exec?"!}
\end{protos}
\noindent
\code{Close-on-exec?} returns true if \code{port} will be closed
 when a new program is exec'ed.
\code{Set-close-on-exec?!} sets \code{port}'s close-on-exec flag.

%; Status flags
%; fcntl(fd, F_GETFL)
%; fcntl(fd, F_SETFL, flags)

\begin{protos}
\proto{i/o-flags}{ port}{file-options}
\protonoresultnoindex{set-i/o-flags!}{ port file-options}\mainschindex{set-i/o-flags"!}
\end{protos}
\noindent
These two procedures read and write various options for \code{port}.
The options that can be read are \code{append}, \code{nonblocking},
 \code{read-only}, \code{write-only}, and \code{read/write}.
Only the \code{append} and \code{nonblocking} can be written.

% off_t lseek(int fd, off_t offset, int whence)

%\subsection{Terminals}

\begin{protos}
\proto{port-is-a-terminal?}{ port}{boolean}
\proto{port-terminal-name}{ port}{string}
\end{protos}
\noindent
\code{Port-is-a-terminal?} returns true if \cvar{port} has an underlying
 file descriptor that is associated with a terminal.
For such ports \code{port-terminal-name} returns the name of the
 terminal, for all others it returns \code{\#f}.

% 6. File Synchronization
%
% int fsync(int fd)              % optional
% int fdatasync(int fd)          % optional
%
% 7. Asynchronous Input and Output
%
% All optional

\section{Regular expressions}
% Regexps - where is the C interface defined? (In part 2, the shell book)

The procedures in this section provide access to POSIX regular expression
 matching.
The regular expression syntax and semantics are far too complex to
 be described here.
Due to limitations in the underlying facility, only Latin-1 strings
are guaranteed to work here---on some platforms, only ASCII may
function correctly.
Moreover, because the C interface uses zero bytes for marking the ends of strings,
 patterns and strings that contain zero bytes will not work
 correctly.

These procedures are in structures \code{posix-regexps} and \code{posix}.

An abstract data type for creating POSIX regular expressions is
 described in section~\ref{regexp-adt}.

\begin{protos}
\proto{make-regexp}{ string . regexp-options}{regexp}
\syntaxproto{regexp-option}{ \cvar{option-name}}{regexp-option}
\end{protos}
\noindent
\code{Make-regexp} makes a new regular expression, using \cvar{string}
 as the pattern.
The possible option names are:

\begin{center}
\begin{tabular}{ll}
\code{extended} & use the extended patterns \\
\code{ignore-case} & ignore case when matching \\
\code{submatches} & report submatches \\
\code{newline} & treat newlines specially
\end{tabular}
\end{center}

The regular expression is not compiled until it matched against a string,
 so any errors in the pattern string will not be reported until that
 point.

\begin{protos}
\proto{regexp?}{ x}{boolean}
\end{protos}
\noindent
This is a predicate for regular expressions.

\begin{protos}
\protonoresult{regexp-match}{ regexp string start submatches? starts-line? ends-line?}
\protoresult{boolean or list of matches}
\proto{match?}{ x}{boolean}
\proto{match-start}{ match}{integer}
\proto{match-end}{ match}{integer}
\end{protos}
\noindent
\code{Regexp-match} matches the regular expression against the characters
 in \cvar{string}, starting at position \cvar{start}.
If the string does not match the regular expression, \code{regexp-match}
 returns \code{\#f}.
If the string does match, then a list of match records is returned
 if \cvar{submatches?} is true, or \code{\#t} is returned if it is not.
Each match record contains the index of the character at the beginning
 of the match and one more than the index of the character at the end.
The first match record gives the location of the substring that matched
 \cvar{regexp}.
If the pattern in \cvar{regexp} contained submatches, then the results
 of these are returned in order, with a match records reporting submatches
 that succeeded and \code{\#f} in place of those that did not.

\cvar{Starts-line?} should be true if \cvar{string} starts at the beginning
 of a line and \cvar{ends-line?} should be true if it ends one.

\section{Syslog facility}
\label{syslog-facility}

The procedures in this section provide access to the POSIX syslog
facility.  The functionality is in a structure called
\code{posix-syslog}.  The Scheme~48 interface to the
syslog facility differs significantly from that of the Unix library
functionality in order to support multiple simultaneous connections to
the syslog facility.

Log messages carry a variety of parameters beside the text of the
message itself, namely a set of options controlling the output format
and destination, the facility identifying the class of programs the
message is coming from, an identifier specifying the concrete program,
and the level identifying the importance of the message.  Moreover, a
log mask can prevent messages at certain levels to be actually sent to
the syslog daemon.

\subsection*{Log options}

A log option specifies details of the I/O behavior of the syslog
facility.  A syslog option is an element of a finite type (see
Section~\ref{sec:finite-types}) constructed by the
\code{syslog-option} macro.  The syslog facility works with sets of
options which are represented as enum sets (see
Section~\ref{sets-finite-types}).

\begin{protos}
\syntaxproto{syslog-option}{ \cvar{option-name}}{option}
\proto{syslog-option?}{ x}{boolean}
\proto{make-syslog-options}{ list}{options}
\syntaxproto{syslog-options}{ \cvar{option-name} \ldots}{options}
\proto{syslog-options?}{ x}{boolean}
\end{protos}
\noindent
\code{Syslog-option} constructs a log option from the name of an
option.  (The possible names are listed below.)  \code{Syslog-option?}
is a predicate for log options.  Options are comparable using
\code{eq?}.  \code{Make-syslog-options} constructs a set of options
from a list of options.  \code{Syslog-options} is a macro which
expands into an expression returning a set of options from names.
\code{Syslog-options?} is a predicate for sets of options.

Here is a list of possible names of syslog options:

\begin{description}
\item[\code{console}]
  If syslog cannot pass the message to syslogd it will attempt to
  write the message to the console.

\item[\code{delay}]
  Delay opening the connection to syslogd immediately until the first
  message is logged.

\item[\code{no-delay}]
  Open the connection to syslogd immediately.  Normally
  the open is delayed until the first message is logged.
  Useful for programs that need to manage the order in which
  file descriptors are allocated.

  \noindent\textbf{NOTA BENE:}
  The \code{delay} and \code{no-delay} options are included for
  completeness, but do not have the expected effect in the present
  Scheme interface: Because the Scheme interface has to multiplex
  multiple simultaneous connections to the syslog facility over a
  single one, open and close operations on that facility happen at
  unpredictable times.

\item[\code{log-pid}]
  Log the process id with each message: useful for identifying
  instantiations of daemons.

\item[\code{no-wait}]
  Do not wait for child processes.
\end{description}

\subsection*{Log facilities}

A log facility identifies the originator of a log message from a
finite set known to the system.  Each originator is identified by a
name:

\begin{protos}
\syntaxproto{syslog-facility}{ \cvar{facility-name}}{facility}
\proto{syslog-facility?}{ x}{boolean}
\end{protos}
\noindent
\code{Syslog-facility} is macro that expands into an expression
returning a facility for a given name.  \code{Syslog-facility?} is a
predicate for facilities.  Facilities are comparable via \code{eq?}.

Here is a list of possible names of syslog facilities:

\begin{description}
\item[\code{authorization}]
  The authorization system: login, su, getty, etc.

\item[\code{cron}]
  The cron daemon.

\item[\code{daemon}]
  System daemons, such as routed, that are not provided for explicitly
  by other facilities.

\item[\code{kernel}]
  Messages generated by the kernel.

\item[\code{lpr}]
  The line printer spooling system: lpr, lpc, lpd, etc.

\item[\code{mail}]
  The mail system.

\item[\code{news}]
  The network news system.

\item[\code{user}]
  Messages generated by random user processes.

\item[\code{uucp}]
  The uucp system.

\item[\code{local0} \code{local1} \code{local2} \code{local3} \code{local4} \code{local5} \code{local6} \code{local7}]
  Reserved for local use.
\end{description}

\subsection*{Log levels}

A log level identifies the importance of a message from a fixed set
of possible levels.

\begin{protos}
\syntaxproto{syslog-level}{ \cvar{level-name}}{level}
\proto{syslog-level?}{ x}{boolean}
\end{protos}
\noindent
\code{Syslog-level} is macro that expands into an expression returning
a facility for a given name.  \code{Syslog-level?} is a predicate for
facilities.  Levels are comparable via \code{eq?}.

Here is a list of possible names of syslog levels:

\begin{description}
\item[\code{emergency}]
  A panic condition.  This is normally broadcast to all users.

\item[\code{alert}]
  A condition that should be corrected immediately, such as a
  corrupted system database.

\item[\code{critical}]
  Critical conditions, e.g., hard device errors.

\item[\code{error}]
  Errors.

\item[\code{warning}]
  Warning messages.

\item[\code{notice}]
  Conditions that are not error conditions, but should possibly be
  handled specially.

\item[\code{info}]
  Informational messages.

\item[\code{debug}]
  Messages that contain information normally of use only when
  debugging a program.
\end{description}

\subsection*{Log masks}

A log masks can mask out log messages at a set of levels.  A log
mask is an enum set of log levels.

\begin{protos}
\proto{make-syslog-mask}{ list}{mask}
\syntaxproto{syslog-mask}{ \cvar{level-name} \ldots}{mask}
\constproto{syslog-mask-all}{ mask}
\proto{syslog-mask-upto}{ level}{mask}
\proto{syslog-mask?}{ x}{boolean}
\end{protos}
\noindent
\code{Make-syslog-mask} constructs a mask from a list of levels.
\code{Syslog-mask} is a macro which constructs a mask from names of
levels.  \code{Syslog-mask-all} is a predefined log mask containing
all levels.  \code{Syslog-mask-upto} returns a mask consisting of all
levels up to and including a certain level, starting with
\code{emergency}.

\subsection*{Logging}

Scheme~48 dynamically maintains implicit connections to the syslog
facility specifying a current identifier, current options, a current
facility and a current log mask.  Every thread
maintains it own implicit connection to syslog.  Note that the
connection is not implicitly preserved across a \code{spawn}.

\begin{protos}
\proto{with-syslog-destination}{ string options facility mask thunk}{value}
\end{protos}
\noindent
\code{With-syslog-destination} dynamically binds parameters of the
implicit connection to the syslog facility and runs \var{thunk}
within those parameter bindings, returning what \var{thunk}
returns.  Each of the parameters may be \code{\#f} in which case the
previous values will be used.

\begin{protos}
\protonoresult{syslog}{ level message}
\protonoresult{syslog}{ level message string}
\protonoresult{syslog}{ level message string options}
\protonoresult{syslog}{ level message string options syslog-facility}
\protonoresult{syslog}{ level message channel}
\end{protos}
\noindent
\code{Syslog} actually logs a message.  Each of the parameters of the
implicit connection (except for the log mask) can be explicitly
specified as well for the current call to \code{syslog}, overriding
the parameters of the channel.  The parameters revert to their
original values after the call.

The final form specifies the destination of the log message as a
channel; see the next section.

\subsection*{Syslog channels}
%
These procedures allow direct manipulation of syslog channels, the
objects that represent connections to the syslog facility.  Note that
it is not necessary to explicitly open a syslog channel to do logging.

\begin{protos}
\proto{open-syslog-channel}{string options facility mask}{channel}
\protonoresult{close-syslog-channel}{channel}
\proto{with-syslog-destination}{ channel thunk}{value}
\end{protos}
\noindent
\code{Open-syslog-channel} and \code{close-syslog-channel} create and
destroy a connection to the syslog facility, respectively.  The
specified form of calling \code{syslog} logs to the specified channel.

\code{With-syslog-channel} dynamically binds parameters of the
implicit connection to the syslog facility to those specified in
\cvar{channel} and runs \var{thunk} within those parameter bindings,
returning what \var{thunk} returns.

\section{Error codes}

POSIX functions report the nature of an error via \textit{system error
  numbers}---OS-specific integers that encode a variety of different
error situations.  At the Scheme level, error numbers are represented
as \textit{errnos}, objects that specify a name for the error
sitation.  Errnos are definted in the structures \code{posix-errnos}
and \code{posix}.

Currently, the system reports such error situations by raising
exceptions with condition type \code{\&os-error}.  The
\code{\&os-error} condition type has a field \code{code} that contains
the system error number.

There are two varieties of errnos available, {\em named} and {\em anonymous}.
A named errno is one for which we have a symbolic name, such as \code{fault}
 or \code{intr}.
Anonymous errnos, for which we only have the current operating system's
 errno number, have no meaning in other operating systems.
Named errnos preserve their meaning in image files.
Not all named errnos are available from all OS's and
 there may be multiple names for a single OS errno number.

\begin{protos}
\syntaxproto{errno}{ \cvar{errno-name}}{errno}
\proto{name->errno}{ symbol}{errno or {\tt \#f}}
\proto{integer->errno}{ integer}{errno}
\proto{errno?}{ x}{boolean}
\proto{errno-name}{ errno}{symbol or {\tt \#f}}
\proto{errno-os-number}{ errno}{integer}
\proto{errno=?}{ errno0 errno1}{boolean}
\end{protos}
\noindent
The syntax \code{errno} returns a (named) errno associated with
\cvar{errno-name}.
\code{Name->errno} returns a (named) errno or \code{\#f} if the
 the errno \cvar{name} is not supported by the operating system.
The errno returned by \code{integer->errno} is a named errno if
 \cvar{integer} corresponds to a named errno in the current operating
 system; otherwise it returns an anonymous errno.
\code{Errno-name} returns a symbol if \cvar{errno} is named and
 \code{\#f} if it is anonymous.
\code{Errno=?} returns \code{\#t} if \cvar{errno0} and \cvar{errno1}
 have the same operating system number and \code{\#f} if they do not.

\subsection{POSIX errnos}
The following lists the names of the POSIX errnos.
\begin{center}
\begin{longtable}{@{\ttfamily}lp{0.6\textwidth}}
   toobig			     & Argument list too long.\\
   acces				& Permission denied.\\
   addrinuse				& Address in use.\\
   addrnotavail			& Address not available.\\
   afnosupport		       & Address family not supported.\\
   again & Resource unavailable, try again.\\
   already			     & Connection already in progress.\\
   badf				& Bad file descriptor.\\
   badmsg				& Bad message.\\
   busy				& Device or resource busy.\\
   canceled				& Operation canceled.\\
   child				& No child processes.\\
   connaborted			& Connection aborted.\\
   connrefused			& Connection refused.\\
   connreset				& Connection reset.\\
   deadlk			      & Resource deadlock would occur.\\
   destaddrreq		       & Destination address required.\\
   dom	     & Mathematics argument out of domain of function.\\
   dquot				& Reserved.\\
   exist				& File exists.\\
   fault				& Bad address.\\
   fbig				& File too large.\\
   hostunreach			& Host is unreachable.\\
   idrm				& Identifier removed.\\
   ilseq				& Illegal byte sequence.\\
   inprogress				& Operation in progress.\\
   intr				& Interrupted function.\\
   inval				& Invalid argument.\\
   io					& I/O error.\\
   isconn				& Socket is connected.\\
   isdir				& Is a directory.\\
   loop			  & Too many levels of symbolic links.\\
   mfile				& Too many open files.\\
   mlink				& Too many links.\\
   msgsize				& Message too large.\\
   multihop				& Reserved.\\
   nametoolong			& Filename too long.\\
   netdown				& Network is down.\\
   netreset			      & Connection aborted by network.\\
   netunreach				& Network unreachable.\\
   nfile			      & Too many files open in system.\\
   nobufs				& No buffer space available.\\
   nodata & No message is available on the STREAM head read queue. \\
   nodev				& No such device.\\
   noent				& No such file or directory.\\
   noexec			       & Executable file format error.\\
   nolck				& No locks available.\\
   nolink				& Reserved.\\
   nomem				& Not enough space.\\
   nomsg			     & No message of the desired type.\\
   noprotoopt				& Protocol not available.\\
   nospc				& No space left on device.\\
   nosr				& No STREAM resources. \\
   nostr				&  Not a STREAM. \\
   nosys				& Function not supported.\\
   notconn				& The socket is not connected.\\
   notdir				& Not a directory.\\
   notempty				& Directory not empty.\\
   notsock				& Not a socket.\\
   notsup				& Not supported.\\
   notty			& Inappropriate I/O control operation.\\
   nxio				& No such device or address.\\
   opnotsupp			  & Operation not supported on socket.\\
   overflow		  & Value too large to be stored in data type.\\
   perm				& Operation not permitted.\\
   pipe				& Broken pipe.\\
   proto				& Protocol error.\\
   protonosupport			& Protocol not supported.\\
   prototype			     & Protocol wrong type for socket.\\
   range				& Result too large.\\
   rofs				& Read-only file system.\\
   spipe				& Invalid seek.\\
   srch				& No such process.\\
   stale				& Reserved.\\
   time			     & Stream ioctl() timeout. \\
   timedout				& Connection timed out.\\
   txtbsy				& Text file busy.\\
   wouldblock	& Operation would block.\\
   xdev				& Cross-device link.\\
\end{longtable}
\end{center}

\section{C to Scheme correspondence}
\label{function-correspondence}

The following table lists the Scheme procedures that correspond to
 particular C procedures.
Not all of the Scheme procedures listed are part of the POSIX interface.

\texonly\setlongtables
\begin{longtable}{ll}\endtexonly
\htmlonly\begin{tabular}{ll}\endhtmlonly
C procedure & Scheme procedure(s) \\
\endfirsthead
 C procedure & Scheme procedure(s) \\
\endhead
\code{access}&\code{accessible?}\\
\code{asctime}&\code{date->string}\\
\code{chdir}&\code{set-working-directory!}\\
\code{close}&\code{close-input-port, close-output-port,}\\
 &\code{\ \ close-channel, close-socket}\\
\code{closedir}&\code{close-directory-stream}\\
\code{creat}&\code{open-file}\\
\code{ctime}&\code{time->string}\\
\code{dup}&\code{dup, dup-switching-mode}\\
\code{dup2}&\code{dup2}\\
\code{exec[l|v][e|p|$\epsilon$]}
 &\code{exec, exec-with-environment,} \\
 &\code{\ \ exec-file, exec-file-with-environment,} \\
 &\code{\ \ exec-with-alias}\\
\code{\_exit}&\code{exit}\\
\code{fcntl}&\code{io-flags, set-io-flags!,}\\
 &\code{\ \ close-on-exec, set-close-on-exec!}\\
\code{fork}&\code{fork, fork-and-forget}\\
\code{fstat}&\code{get-port-info}\\
\code{getcwd}&\code{working-directory}\\
\code{getegid}&\code{get-effective-group-id}\\
\code{getenv}&\code{lookup-environment-variable,} \\
 &\code{\ \ environment-alist}\\
\code{geteuid}&\code{get-effective-user-id}\\
\code{getgid}&\code{get-group-id}\\
\code{getgroups}&\code{get-groups}\\
\code{getlogin}&\code{get-login-name}\\
\code{getpid}&\code{get-process-id}\\
\code{getppid}&\code{get-parent-process-id}\\
\code{getuid}&\code{get-user-id}\\
\code{gmtime}&\code{time->local-date}\\
\code{isatty}&\code{port-is-a-terminal?}\\
\code{link}&\code{link}\\
\code{localtime}&\code{time->utc-date}\\
\code{lstat}&\code{get-file/link-info}\\
\code{mkdir}&\code{make-directory}\\
\code{mkfifo}&\code{make-fifo}\\
\code{mktime}&\code{date->time}\\
\code{open}&\code{open-file}\\
\code{opendir}&\code{open-directory-stream}\\
\code{pipe}&\code{open-pipe}\\
\code{read}&\code{read-char, read-block}\\
\code{readdir}&\code{read-directory-stream}\\
\code{readlink}&\code{read-symbolic-link}\\
\code{rename}&\code{rename}\\
\code{rmdir}&\code{remove-directory}\\
\code{setgid}&\code{set-group-id!}\\
\code{setuid}&\code{set-user-id!}\\
\code{stat}&\code{get-file-info}\\
\code{strftime}&\code{format-date}\\
\code{symlink}&\code{create-symbolic-link}\\
\code{syslog}&\code{syslog}\\
\code{time}&\code{current-time}\\
\code{ttyname}&\code{port-terminal-name}\\
\code{umask}&\code{set-file-creation-mask!}\\
\code{uname}&\code{os-name, os-node-name,} \\
 &\code{\ \ os-release-name, os-version-name,} \\
 &\code{\ \ machine-name}\\
\code{unlink}&\code{unlink}\\
\code{waitpid}&\code{wait-for-child-process}\\
\code{write}&\code{write-char, write-block}\\
% The ones that aren't there yet.
%wait \\
%kill \\
%sigemptyset, sigfillset, sigaddset, digdelset, sigismember \\
%sigaction \\
%sigsuspend \\
%alarm \\
%pause \\
%sleep \\
%getpgrp \\
%setsid \\
%times \\
%ctermid \\
%sysconf \\
%rewinddir \\
%chmod \\
%chown \\
%utime \\
%ftruncate \\
%pathconf, fpathconf \\
%lseek \\
%fsync \\
\texonly\end{longtable}\endtexonly
\htmlonly\end{tabular}\endhtmlonly

%%% Local Variables:
%%% mode: latex
%%% TeX-master: "manual"
%%% End:
